\section{Introduction}

\subsection{Working title}
\textit{Design and implementation of the Meta Casanova 3 compiler back-end}

\subsection{Motive}
Kenniscentrum is interested in innovative technologies.
Innovative technologies like virtual reality and video games are the fields our research group is researching.

In order to ease the development of virtual reality and video games, the Casanova language was developed.
The Casanova language is the subject of the PhD thesis of Francesco.

The complex nature of the Casanova language lead to a complex compiler.
To simplify the development of Casanova, the language Meta Casanova was developed.

\subsection{Importance}
The resulting program will be used by the research group to build the meta Casanova 3 compiler, and the resulting thesis will be used as documentation for the future developers of MC.

This will be useful to the research group.

\subsection{Goal}
The goal of the assignment is to have a working back-end that is able to produce an executable.

\subsection{Problem statement}
At the start of the assignment, there is only a definition of MC 2.
At the end of the assignment, there is a working back-end for MC 3.

\subsection{Research question}
The primary research question of this thesis is:

\textit{How to implement a transformation from typechecked Meta Casanova (MC) from the front-end, to executable code within the timeframe of the internship?}

Where the transformation must satisfy these requirements:
\begin{description}
    \item[The correctness requirement] The back-end must in no case produce an incorrect program.
    \item[The .NET requirement] The executable must be able to inter-operate with .NET.
    \item[The multiplatform requirement] The generated code must run on all the platforms .NET runs on.
    \item[The performance requirement] The performance of the generated program should be better than Python.
\end{description}

The correctness requirement exists because the compiler must be reliable.
Any program can at most be as reliable as the compiler used to generate it.
\label{whydotnet}
The .NET requirement exists because of the need for a large library and inter-operability with Unity game engine.
This is because the main area of research of the organization is game-related\footnote{see section~\ref{motive}}.
The multiplatform requirement is because the games are produced for any platform.
The performance requirement is there because games have to be fast.

In order to answer the research question, seven sub-questions were formulated.

\begin{enumerate}
    \item In what language should the code generator produce its output?
    \item What should the interface be between the front-end and the back-end?
    \item What should the intermediate representation of the functions be?
    \item How does the interface map to the output language?
    \item How to generate names so that they comply with the output language?
    \item How to validate the code-generator?
    \item How to validate the test programs?
\end{enumerate}

Each answer of a sub-question is provided evidence by implementing a part of the back-end. 
This will in turn provide evidence to answer the main research question.

\subsection{client}
The graduation assignment is carried out at Kenniscentrum Creating 010.
\textit{Kenniscentrum Creating 010 is a transdisciplinary design-inclusive Research Center enabling citizens, students and creative industry making the future of Rotterdam}~\cite{creating2016home}.

The research group is creating the Casanova language.
The members of the research group are 
  Francesco di Giacomo\footnote{\label{venice}Universita' Ca' Foscari, Venezia}, 
  Mohamed Abbadi\footnoteref{venice}, 
  Agostino Cortesi\footnoteref{venice}, 
  Giuseppe Maggiore\footnote{Hogeschool Rotterdam} and 
  Pieter Spronck\footnote{Tilburg University}.

Within the research group is our research team, tasked with the design and implementation of Meta Casanova.
The research team is supervised by Giuseppe Maggiore and comprises of three students.
Louis van der Burg, responsible for developing the Meta Casanova language,
Jarno Holstein, responsible for the front-end of the Meta Casanova compiler,
and Douwe van Gijn, responsible for the back-end of the Meta Casanova compiler.

\subsection{Workplace and tasks}
During the internship the student will be part of a research team that develops and implements the MC compiler.
The student periodically relay the progress and receive feedback every two weeks.
The student will be supported by the research group.

